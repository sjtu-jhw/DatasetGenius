\section{Phylogenetic Trees}\label{sec:trees-of-life}

One of the central tenets following from the theory of organismal
evolution is that all life is related through descent with
modification \cite{Darwin1859-ka}. That is, populations of a
biological species can over time diverge enough, due to natural
selection, adaptation, genetic drift, and other forces acting
differentially on different populations, that they form new species,
some of which persist and go on themselves to split, giving rise to
new species, and so forth. Speciation through diversification can
sometimes be driven by new ecologic opportunities, for example when
new habitats are being colonized, a process often referred to as
adaptive radiation \cite{Simpson1949-pv,Simpson1953-qg}. One of the
most prominent research objectives in evolutionary science is to
reconstruct, using genetic and organismal trait data, the evolutionary
history of different organisms, species, or life forms; i.e., to
reconstruct the lines of shared descent by which organisms are
connected \cite{Felsenstein2003-ji,Swofford1996-ya}. Such a
reconstruction is represented in the form of a phylogenetic tree, in
which the leaves are often called operational taxonomic units (OTUs)
and represent the sampled entities, and internal nodes represent
ancestral entities, such as ancestral populations from which
descendent ones diverged. Phylogenetic reconstruction results in
unrooted trees; the root is normally not known (and cannot normally be
sampled), but reasonably accurate mechanisms for inducing a root exist
\cite{Huelsenbeck2002-hm,Maddison1984-qa} (for example, by including
in the reconstruction analysis a group of species -- a so-called
``outgroup'' -- that are already known to fall outside of the ingroup
for which evolutionary patterns are being studied).

A phylogenetic tree represents important evolutionary hypotheses about\linebreak[4]
shared history. For example, two OTUs A and B are more closely related
to each other than to OTU C if A and B share a more recent common
ancestor than they do with C. The subtree descending from a node forms
a clade, clades which share a parent are called sister clades. One of
the major objects of comparative phylogenetics is to identify the
properties and processes (organismal traits, geographic range, tempo and mode of
evolution, etc) by which one clade differs from others, in particular
its sisters, and how these properties change along lines of descent in
the tree \cite{Felsenstein1985-au,OMeara2011-br}. This gives rise to a
number of important queries when mapping data onto phylogenetic trees
for (or as a result of) analysis. Particularly ubiquitous operations
on trees include the following: (1) finding the most recent common
ancestor of a given number of nodes (usually leaf nodes); (2)
enumerating the leaf nodes, or all nodes descending from a given
(internal) node; (3) enumerating the sequence of ancestors of a node
to the root; and (4) identifying the last ancestor of a node A from
which another node B is not also descended. We will come back to these and other operations as part of the competency questions for our modeling in Section \ref{sec:arb-trees}.

Operations (1) and (4) correspond to two principle ways in which the
semantics of clade concepts can be defined on a tree
\cite{De_Queiroz1990-bv}, whether using a concrete instantiation of a
tree, or a hypothetical one. In the field of phylogenetic taxonomy
\cite{De_Queiroz1992-oq}, a clade concept defined by the most recent
common ancestor of a set of (usually leaf) nodes includes the common
ancestor and is referred to as a node-based definition. In contrast, a
branch-based definition circumscribes the clade as the last ancestor
of a (usually leaf) node that excludes (i.e., does not have as a
descendant) another node (also usually a leaf node). The semantics of
a clade concept defined in this way is such that the branch subtending
from the ancestor node to its parent is included (hence the name
“branch-based”). To understand this, remember that a phylogenetic tree
is a model of evolutionary lines of descent reconstructed from sampled
data. In reality, there may be lines of descent which were not
observed (sampled), for example because all organisms from those lines
are now extinct, but which, had they been observed, would originate
from the subtending branch and which would therefore still be included
in the clade because they would branch off after the lineage to be
excluded.


It is worth noting that spurred in part by the exponentially
increasing amount of data available for phylogenetic reconstruction,
very large trees encompassing up to tens of thousands of taxa have
recently become available \cite{Goloboff2009-oa,Driskell2004-wq,Dunn2008-tr,Smith2009-yk,Jarvis2014-lq}, culminating in the initial publication of
the synthesized Open Tree of Life with about 2 million tips \cite{Hinchliff2015-nd}. Such
encompassing trees open up unprecedented opportunities for comparative
phylogenetic research. However, this also means that our knowledge
about the evolution of life is changing at increasing pace and
breadth, which makes it necessary to efficiently map clade definitions
from one tree to another, or from one revision of the Open Tree of
Life to a future one. A recent initiative, termed ``phyloreferencing''
(http://phyloref.org) aims to accomplish this by using machine
reasoning over ontological representations of the semantics of both
clade definitions and phylogenetic trees \cite{Cellinese2015-zv,Michael_Keesey2007-lb,Sereno2005-wv}. In the rest of this paper, we abstract from the specific use case and look at the task of ontological modeling of trees in general.
