\section{Trees With Bounded Arity}\label{sec:bin-trees}

In this section, we look at $n$-bounded trees as a special case, and present a set of OWL axioms that can be employed to model these.

\begin{definition}
A \emph{rooted directed branching $n$-bounded tree} (short: \emph{$n$-bounded tree}) is a tree $T = (V, E)$ where every node has at most $n$ outgoing edges; i.e., for every $v \in V$, $\vert E \cap (\{v\} \times V) \vert \leq n$.
\end{definition}

In the previous section, we have discussed why we needed to include an explicit hasSibling relation in our model, namely because we were unable to axiomatically define it in OWL. If we know that the trees under consideration are $n$-bounded, though, we can in fact infer the hasSibling relation. 

To this end, we introduce a set of additional axioms (see Figure \ref{figure:binaryTreeAxioms}) that, if combined with the axioms  from Figure \ref{figure:tree} can be used to axiomatize the structure of an $n$-bounded tree. Key to this 
are axioms (\ref{disjunction}) and (\ref{disjoint}), which together limit the number of children per node to a maximum of $n$. 

Note that, if these axioms are considered, we can indeed automatically infer the hasSibling relationship: Having an upper bound on the number of children per node, we can use a finite set of concept names $\text{Child}_i$ to differentiate the children of any given node in the tree.
Note how, due to (\ref{disjunction}) and (\ref{disjoint}), every $n$-ary tree node is typed by one and only one class $\text{Child}_i$.
Moreover, axioms in (\ref{atMost1Childi}) enforce that at most one child of every node is in the class $\text{Child}_i$ for every $i = 1, \ldots, n$.
Using axioms in (\ref{selfConnect}), we automatically infer that each child of every node is connected to itself via some property $R_i$ for some $i = 1, \ldots, n$.
Using axiom (\ref{chain}), %(\ref{symmetry}) and (\ref{transitivity}), 
we can infer the hasSibling relation.

%In turn, this allows use to make use of role chains to correctly infer the hasSibling relation (see axioms (\ref{siblingAxiom1}) and (\ref{siblingAxiom2})).
%Due to the use of differently named roles to connect a node with its children, these axioms will not result in the introduction of reflexive connections over the role hasSibling.


\begin{figure}[t]
\begin{align}
n\text{-BoundedTreeNode} &\sqsubseteq \text{TreeNode} \\
n\text{-BoundedTreeNode} &\sqsubseteq \forall \text{hasAncestor}.n\text{-BoundedTreeNode} \\
n\text{-BoundedTreeNode} &\sqsubseteq \forall \text{hasDescendant}.n\text{-BoundedTreeNode} \\
n\text{-BoundedTreeNode} &\sqsubseteq \text{Child}_1 \sqcup \ldots \sqcup \text{Child}_n \label{disjunction} \\
\{\text{Child}_i \sqcap \text{Child}_j &\sqsubseteq \bot \mid 1 \leq i < j \leq n \} \label{disjoint} \\
\{n\text{-BoundedTreeNode} &\sqsubseteq \mathord{\leq} 1 \text{hasChild}.\text{Child}_i \mid 1 \leq i \leq n\} \label{atMost1Childi}\\
n\text{-BoundedTreeNode} &\sqsubseteq \mathord{\leq} n \text{hasChild}.n\text{-BoundedTreeNode} \\
\{\text{Child}_i &\sqsubseteq \exists R_\text{i}.\textsf{Self} \mid 1 \leq i \leq n\} \label{selfConnect} \\
\{R_\text{i} \circ \text{hasParent} \circ \text{hasChild} \circ R_\text{j} &\sqsubseteq \text{hasSibling} \mid 1 \leq i < j \leq n \} \label{chain} 
%\text{hasSibling} &\sqsubseteq \text{hasSibling}^- \label{symmetry} \\
%\text{hasSibling} \circ \text{hasSibling} &\sqsubseteq \text{hasSibling} \label{transitivity}
\end{align}
\caption{Axioms for the $n$-bounded Tree Pattern: These need to be added to the axioms from Figure \ref{figure:tree}}
\label{figure:binaryTreeAxioms}
\end{figure}

Note, though, that hasSibling is non-simple, i.e. the declaration of irreflexivity for hasSibling from Figure \ref{figure:tree} violates regularity. If the ontology shall fall within OWL DL, the irreflexivity axiom should be removed.

Note also, that the approach just spelled out may not be practical for large $n$, as the number of models to be checked, e.g. by a tableaux-based reasoner, will increase exponentially with $n$ due to the disjunction in (\ref{disjunction}).

%Axioms (\ref{symmetry}) and (\ref{transitivity}) are, strictly speaking, not needed, but they help to clarify the hasSibling relationship. It would also make sense to add declare irreflexivity for hasSibling, which would however violate regularity and would thus cause the ontology to be no longer in OWL DL.

%For the sake of simplicity, we have restricted our construction for binary trees.
%Nevertheless, our approach can be generalized to any tree where the outdegree of the nodes is bounded by some $k$.

%\begin{definition}
%An \emph{$n$-ary tree} is a tree $T = (V, E)$ where every node has at most $n$ outgoing edges; i.e., for every $v \in V$, $\vert E \cap (\{v\} \times V) \vert \leq n$.
%\end{definition}

%To generalize the previously presented set of axioms to model $n$-ary trees, we just have to introduce the following changes:
%First, we introduce $n$ different properties $\text{hasChild}_1, \text{hasChild}_2, \ldots$ and include an axiom such as (50) enforcing that every non-leaf node must have at least one successor connected via some $\text{hasChild}_i$ property.
%Then, we declare each one of these properties to be functional (see axioms (51) and (52)) and make all of these properties a subproperty of \text{hasChild}.
%Moreover, all of these properties must be declared pairwise disjoint and all binary tree nodes must be restricted to have at most $n$ children (see axioms (53) and (56), respectively).
%Finally, we can use axioms such as (57) and (58) to automatically infer the \text{hasSibling} relation.
%Note that, to do so we only have to introduce the following axioms to do so: $\text{hasChild}_i^- \circ \text{hadChild}_{i+1} \sqsubseteq \text{hasSibling}$ for all $i = 1, \ldots, n-1$.
%This is because of axioms (59) and (60) which enforce that the relation \text{hasSibling} is both symmetric and transitive.

%\textbf{David could you write this?}

%2 pages

%Discuss sibling modeling using Sebastian's approach (binary labelling of nodes -- see pictures of whiteboard). 

%Note that this can be extended towards n-ary trees (all non-leaf nodes with same number of children). Discuss informally how.