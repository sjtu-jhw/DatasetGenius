\subsection*{Discussion}\label{sec12}
In this study, we propose a real-time power scheduling approach for a provincial power grid with high penetration of renewable energy. Our method is based on a planning-capable reinforcement learning algorithm and performs real-time rolling-horizon joint optimization. This eliminates the differences in the time scales between unit commitment and economic dispatch in traditional staged optimization, breaking away from the pre-calculation mode. This may solve the current dilemma of power scheduling, as our method no longer depends on the long-term renewable energy forecast results.
Our control design attains many of the expectations for a learning-based optimization approach in the power system community. It offers real-time computation, the ability to incorporate ultra-short-term forecasts, robustness to challenging scenarios, and anticipation of future events. These outcomes were made possible by bridging the gaps in capability and infrastructure through the integration of advances in reinforcement learning and electrical engineering. This includes the development of a realistic, accurate simulator and a security-constrained, planning-capable RL algorithm. The results of our real grid simulation demonstrate the potential to optimize more generators and larger power grids in real-time. RL-based methodology provides a promising direction for future power scheduling.

% Our learning framework has the potential to transform future research on power grid scheduling and operation. In the future, it could be used to explore configurations of storage management, demand response, or even the planning of electric vehicle charging. Additionally, it is also possible to use the AlphaZero-like Monte Carlo Tree Search (MCTS) to develop a warning mechanism that alerts operators of potential failures at an early stage. However, implementing this idea requires a server with hundreds of CPU cores, which cannot be achieved under our current experimental conditions. More broadly, our approach is compatible with an adversarial paradigm, in which the scheduling problem is treated as a board game between an attacker who attempts to collapse the power grid by attacking transmission lines and an operator who tries to make the power grid more stable by adjusting generators, etc. This could significantly improve the robustness of GridZero's policy. We hope to achieve these goals in future research.


% \section{Conclusion}\label{sec13}

