\subsection*{Discussion}\label{sec12}
In this study, we propose a real-time power scheduling approach for a provincial power grid with high penetration of renewable energy. Our method is based on a planning-capable reinforcement learning algorithm and performs real-time rolling-horizon joint optimization. This eliminates the differences in the time scales between unit commitment and economic dispatch in traditional staged optimization, breaking away from the pre-calculation mode. This may solve the current dilemma of power scheduling, as our method no longer depends on the long-term renewable energy forecast results.
Our control design attains many of the expectations for a learning-based optimization approach in the power system community. It offers real-time computation, the ability to incorporate ultra-short-term forecasts, robustness to challenging scenarios, and anticipation of future events. These outcomes were made possible by bridging the gaps in capability and infrastructure through the integration of advances in reinforcement learning and electrical engineering. This includes the development of a realistic, accurate simulator and a security-constrained, planning-capable RL algorithm. The results of our real grid simulation demonstrate the potential to optimize more generators and larger power grids in real-time. RL-based methodology provides a promising direction for future power scheduling.




